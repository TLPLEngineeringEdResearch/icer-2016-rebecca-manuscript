% This is "sig-alternate.tex" V2.1 April 2013
% This file should be compiled with V2.5 of "sig-alternate.cls" May 2012
%
% This example file demonstrates the use of the 'sig-alternate.cls'
% V2.5 LaTeX2e document class file. It is for those submitting
% articles to ACM Conference Proceedings WHO DO NOT WISH TO
% STRICTLY ADHERE TO THE SIGS (PUBS-BOARD-ENDORSED) STYLE.
% The 'sig-alternate.cls' file will produce a similar-looking,
% albeit, 'tighter' paper resulting in, invariably, fewer pages.
%
% ----------------------------------------------------------------------------------------------------------------
% This .tex file (and associated .cls V2.5) produces:
%       1) The Permission Statement
%       2) The Conference (location) Info information
%       3) The Copyright Line with ACM data
%       4) NO page numbers
%
% as against the acm_proc_article-sp.cls file which
% DOES NOT produce 1) thru' 3) above.
%
% Using 'sig-alternate.cls' you have control, however, from within
% the source .tex file, over both the CopyrightYear
% (defaulted to 200X) and the ACM Copyright Data
% (defaulted to X-XXXXX-XX-X/XX/XX).
% e.g.
% \CopyrightYear{2007} will cause 2007 to appear in the copyright line.
% \crdata{0-12345-67-8/90/12} will cause 0-12345-67-8/90/12 to appear in the copyright line.
%
% ---------------------------------------------------------------------------------------------------------------
% This .tex source is an example which *does* use
% the .bib file (from which the .bbl file % is produced).
% REMEMBER HOWEVER: After having produced the .bbl file,
% and prior to final submission, you *NEED* to 'insert'
% your .bbl file into your source .tex file so as to provide
% ONE 'self-contained' source file.
%
% ================= IF YOU HAVE QUESTIONS =======================
% Questions regarding the SIGS styles, SIGS policies and
% procedures, Conferences etc. should be sent to
% Adrienne Griscti (griscti@acm.org)
%
% Technical questions _only_ to
% Gerald Murray (murray@hq.acm.org)
% ===============================================================
%
% For tracking purposes - this is V2.0 - May 2012

\documentclass{$documentclass$}

%%%%%%%%%%%%%%%%%%%%%
%%% Use Packages %%%%
%%%%%%%%%%%%%%%%%%%%%

% $if(fontfamily)$
% \usepackage[$for(fontfamilyoptions)$$fontfamilyoptions$$sep$,$endfor$]{$fontfamily$}
% $else$
% \usepackage{lmodern}
% $endif$
% $if(linestretch)$
% \usepackage{setspace}
% \setstretch{$linestretch$}
% $endif$
% \usepackage{amssymb,amsmath}
% \usepackage{ifxetex,ifluatex}
% \usepackage{fixltx2e} % provides \textsubscript
% \ifnum 0\ifxetex 1\fi\ifluatex 1\fi=0 % if pdftex
%   \usepackage[$if(fontenc)$$fontenc$$else$T1$endif$]{fontenc}
%   \usepackage[utf8]{inputenc}
% $if(euro)$
%   \usepackage{eurosym}
% $endif$
% \else % if luatex or xelatex
%   \ifxetex
%     \usepackage{mathspec}
%   \else
%     \usepackage{fontspec}
%   \fi
%   \defaultfontfeatures{Ligatures=TeX,Scale=MatchLowercase}
% $if(euro)$
%   \newcommand{\euro}{€}
% $endif$
% $if(mainfont)$
%     \setmainfont[$for(mainfontoptions)$$mainfontoptions$$sep$,$endfor$]{$mainfont$}
% $endif$
% $if(sansfont)$
%     \setsansfont[$for(sansfontoptions)$$sansfontoptions$$sep$,$endfor$]{$sansfont$}
% $endif$
% $if(monofont)$
%     \setmonofont[Mapping=tex-ansi$if(monofontoptions)$,$for(monofontoptions)$$monofontoptions$$sep$,$endfor$$endif$]{$monofont$}
% $endif$
% $if(mathfont)$
%     \setmathfont(Digits,Latin,Greek)[$for(mathfontoptions)$$mathfontoptions$$sep$,$endfor$]{$mathfont$}
% $endif$
% $if(CJKmainfont)$
%     \usepackage{xeCJK}
%     \setCJKmainfont[$for(CJKoptions)$$CJKoptions$$sep$,$endfor$]{$CJKmainfont$}
% $endif$
% \fi
% use upquote if available, for straight quotes in verbatim environments
% \IfFileExists{upquote.sty}{\usepackage{upquote}}{}
% use microtype if available
% \IfFileExists{microtype.sty}{%
% \usepackage{microtype}
% \UseMicrotypeSet[protrusion]{basicmath} % disable protrusion for tt fonts
% }{}
% $if(geometry)$
% \usepackage[$for(geometry)$$geometry$$sep$,$endfor$]{geometry}
% $endif$
% \usepackage{hyperref}
% $if(colorlinks)$
% \PassOptionsToPackage{usenames,dvipsnames}{color} % color is loaded by hyperref
% $endif$
% \hypersetup{unicode=true,
% $if(title-meta)$
%             pdftitle={$title-meta$},
% $endif$
% $if(author-meta)$
%             pdfauthor={$author-meta$},
% $endif$
% $if(keywords)$
%             pdfkeywords={$for(keywords)$$keywords$$sep$; $endfor$},
% $endif$
% $if(colorlinks)$
%             colorlinks=true,
%             linkcolor=$if(linkcolor)$$linkcolor$$else$Maroon$endif$,
%             citecolor=$if(citecolor)$$citecolor$$else$Blue$endif$,
%             urlcolor=$if(urlcolor)$$urlcolor$$else$Blue$endif$,
% $else$
%             pdfborder={0 0 0},
% $endif$
%             breaklinks=true}
% \urlstyle{same}  % don't use monospace font for urls
% $if(lang)$
% \ifnum 0\ifxetex 1\fi\ifluatex 1\fi=0 % if pdftex
%   \usepackage[shorthands=off,$for(babel-otherlangs)$$babel-otherlangs$,$endfor$main=$babel-lang$]{babel}
% $if(babel-newcommands)$
%   $babel-newcommands$
% $endif$
% \else
%   \usepackage{polyglossia}
%   \setmainlanguage[$polyglossia-lang.options$]{$polyglossia-lang.name$}
% $for(polyglossia-otherlangs)$
%   \setotherlanguage[$polyglossia-otherlangs.options$]{$polyglossia-otherlangs.name$}
% $endfor$
% \fi
% $endif$
% $if(natbib)$
% \usepackage{natbib}
% \bibliographystyle{$if(biblio-style)$$biblio-style$$else$plainnat$endif$}
% $endif$
% $if(biblatex)$
% \usepackage$if(biblio-style)$[style=$biblio-style$]$endif${biblatex}
% $if(biblatexoptions)$\ExecuteBibliographyOptions{$for(biblatexoptions)$$biblatexoptions$$sep$,$endfor$}$endif$
% $for(bibliography)$
% \addbibresource{$bibliography$}
% $endfor$
% $endif$
% $if(listings)$
% \usepackage{listings}
% $endif$
% $if(lhs)$
% \lstnewenvironment{code}{\lstset{language=Haskell,basicstyle=\small\ttfamily}}{}
% $endif$
% $if(highlighting-macros)$
% $highlighting-macros$
% $endif$
% $if(verbatim-in-note)$
% \usepackage{fancyvrb}
% \VerbatimFootnotes % allows verbatim text in footnotes
% $endif$
% $if(tables)$
% \usepackage{longtable,booktabs}
% $endif$
% $if(graphics)$
% \usepackage{graphicx,grffile}
% \makeatletter
% \def\maxwidth{\ifdim\Gin@nat@width>\linewidth\linewidth\else\Gin@nat@width\fi}
% \def\maxheight{\ifdim\Gin@nat@height>\textheight\textheight\else\Gin@nat@height\fi}
% \makeatother
% Scale images if necessary, so that they will not overflow the page
% margins by default, and it is still possible to overwrite the defaults
% using explicit options in \includegraphics[width, height, ...]{}
% \setkeys{Gin}{width=\maxwidth,height=\maxheight,keepaspectratio}
% $endif$
% $if(links-as-notes)$
% % Make links footnotes instead of hotlinks:
% \renewcommand{\href}[2]{#2\footnote{\url{#1}}}
% $endif$
% $if(strikeout)$
% \usepackage[normalem]{ulem}
% % avoid problems with \sout in headers with hyperref:
% \pdfstringdefDisableCommands{\renewcommand{\sout}{}}
% $endif$
% % $if(indent)$
% % $else$
% % \IfFileExists{parskip.sty}{%
% % \usepackage{parskip}
% % }{% else
% % \setlength{\parindent}{0pt}
% % \setlength{\parskip}{6pt plus 2pt minus 1pt}
% % }
% % $endif$
% \setlength{\emergencystretch}{3em}  % prevent overfull lines
% \providecommand{\tightlist}{%
%   \setlength{\itemsep}{0pt}\setlength{\parskip}{0pt}}
% $if(numbersections)$
% \setcounter{secnumdepth}{5}
% $else$
% \setcounter{secnumdepth}{0}
% $endif$
% $if(subparagraph)$
% $else$
% % Redefines (sub)paragraphs to behave more like sections
% \ifx\paragraph\undefined\else
% \let\oldparagraph\paragraph
% \renewcommand{\paragraph}[1]{\oldparagraph{#1}\mbox{}}
% \fi
% \ifx\subparagraph\undefined\else
% \let\oldsubparagraph\subparagraph
% \renewcommand{\subparagraph}[1]{\oldsubparagraph{#1}\mbox{}}
% \fi
% $endif$
% $if(dir)$
% \ifxetex
%   % load bidi as late as possible as it modifies e.g. graphicx
%   $if(latex-dir-rtl)$
%   \usepackage[RTLdocument]{bidi}
%   $else$
%   \usepackage{bidi}
%   $endif$
% \fi
% \ifnum 0\ifxetex 1\fi\ifluatex 1\fi=0 % if pdftex
%   \TeXXeTstate=1
%   \newcommand{\RL}[1]{\beginR #1\endR}
%   \newcommand{\LR}[1]{\beginL #1\endL}
%   \newenvironment{RTL}{\beginR}{\endR}
%   \newenvironment{LTR}{\beginL}{\endL}
% \fi


$if(title)$
\maketitle
$endif$

% $for(include-before)$
% $include-before$
%
% $endfor$
% Copyright
\setcopyright{acmcopyright}
%\setcopyright{acmlicensed}
%\setcopyright{rightsretained}
%\setcopyright{usgov}
%\setcopyright{usgovmixed}
%\setcopyright{cagov}
%\setcopyright{cagovmixed}


% DOI
% \doi{10.475/123_4}

% ISBN
% \isbn{123-4567-24-567/08/06}

%Conference
% \conferenceinfo{PLDI '13}{June 16--19, 2013, Seattle, WA, USA}

% \acmPrice{15.00}

%
% --- Author Metadata here ---
% \conferenceinfo{WOODSTOCK}{'97 El Paso, Texas USA}
%\CopyrightYear{2007} % Allows default copyright year (20XX) to be over-ridden - IF NEED BE.
%\crdata{0-12345-67-8/90/01}  % Allows default copyright data (0-89791-88-6/97/05) to be over-ridden - IF NEED BE.
% --- End of Author Metadata ---

% \title{Alternate {\ttlit ACM} SIG Proceedings Paper in LaTeX
% Format\titlenote{(Produces the permission block, and
% copyright information). For use with
% SIG-ALTERNATE.CLS. Supported by ACM.}}
% \subtitle{[Extended Abstract]
% \titlenote{A full version of this paper is available as
% \textit{Author's Guide to Preparing ACM SIG Proceedings Using BibTeX} at
% \texttt{www.acm.org/eaddress.htm}}}
%
% You need the command \numberofauthors to handle the 'placement
% and alignment' of the authors beneath the title.
%
% For aesthetic reasons, we recommend 'three authors at a time'
% i.e. three 'name/affiliation blocks' be placed beneath the title.
%
% NOTE: You are NOT restricted in how many 'rows' of
% "name/affiliations" may appear. We just ask that you restrict
% the number of 'columns' to three.
%
% Because of the available 'opening page real-estate'
% we ask you to refrain from putting more than six authors
% (two rows with three columns) beneath the article title.
% More than six makes the first-page appear very cluttered indeed.
%
% Use the \alignauthor commands to handle the names
% and affiliations for an 'aesthetic maximum' of six authors.
% Add names, affiliations, addresses for
% the seventh etc. author(s) as the argument for the
% \additionalauthors command.
% These 'additional authors' will be output/set for you
% without further effort on your part as the last section in
% the body of your article BEFORE References or any Appendices.

\numberofauthors{8} %  in this sample file, there are a *total*
% of EIGHT authors. SIX appear on the 'first-page' (for formatting
% reasons) and the remaining two appear in the \additionalauthors section.
%
\author{
% You can go ahead and credit any number of authors here,
% e.g. one 'row of three' or two rows (consisting of one row of three
% and a second row of one, two or three).
%
% The command \alignauthor (no curly braces needed) should
% precede each author name, affiliation/snail-mail address and
% e-mail address. Additionally, tag each line of
% affiliation/address with \affaddr, and tag the
% e-mail address with \email.
%
% 1st. author
\alignauthor
Ben Trovato\titlenote{Dr.~Trovato insisted his name be first.}\\
       \affaddr{Institute for Clarity in Documentation}\\
       \affaddr{1932 Wallamaloo Lane}\\
       \affaddr{Wallamaloo, New Zealand}\\
       \email{trovato@corporation.com}
% 2nd. author
\alignauthor
G.K.M. Tobin\titlenote{The secretary disavows
any knowledge of this author's actions.}\\
       \affaddr{Institute for Clarity in Documentation}\\
       \affaddr{P.O. Box 1212}\\
       \affaddr{Dublin, Ohio 43017-6221}\\
       \email{webmaster@marysville-ohio.com}
% 3rd. author
\alignauthor Lars Th{\o}rv{\"a}ld\titlenote{This author is the
one who did all the really hard work.}\\
       \affaddr{The Th{\o}rv{\"a}ld Group}\\
       \affaddr{1 Th{\o}rv{\"a}ld Circle}\\
       \affaddr{Hekla, Iceland}\\
       \email{larst@affiliation.org}
\and  % use '\and' if you need 'another row' of author names
% 4th. author
\alignauthor Lawrence P. Leipuner\\
       \affaddr{Brookhaven Laboratories}\\
       \affaddr{Brookhaven National Lab}\\
       \affaddr{P.O. Box 5000}\\
       \email{lleipuner@researchlabs.org}
% 5th. author
\alignauthor Sean Fogarty\\
       \affaddr{NASA Ames Research Center}\\
       \affaddr{Moffett Field}\\
       \affaddr{California 94035}\\
       \email{fogartys@amesres.org}
% 6th. author
\alignauthor Charles Palmer\\
       \affaddr{Palmer Research Laboratories}\\
       \affaddr{8600 Datapoint Drive}\\
       \affaddr{San Antonio, Texas 78229}\\
       \email{cpalmer@prl.com}
}
% There's nothing stopping you putting the seventh, eighth, etc.
% author on the opening page (as the 'third row') but we ask,
% for aesthetic reasons that you place these 'additional authors'
% in the \additional authors block, viz.
% \additionalauthors{Additional authors: John Smith (The Th{\o}rv{\"a}ld Group,
% email: {\texttt{jsmith@affiliation.org}}) and Julius P.~Kumquat
% (The Kumquat Consortium, email: {\texttt{jpkumquat@consortium.net}}).}
% \date{30 July 1999}
% Just remember to make sure that the TOTAL number of authors
% is the number that will appear on the first page PLUS the
% number that will appear in the \additionalauthors section.

\maketitle

$if(abstract)$
\begin{abstract}
  $abstract$
\end{abstract}
$endif$
%
% The code below should be generated by the tool at
% http://dl.acm.org/ccs.cfm
% Please copy and paste the code instead of the example below.
%
\begin{CCSXML}

\end{CCSXML}

%
%  Use this command to print the description
%
\printccsdesc

% We no longer use \terms command
%\terms{Theory}

\keywords{ACM proceedings; \LaTeX; text tagging}

% Print the body of the document
$body$
% Handle References


\end{document}
