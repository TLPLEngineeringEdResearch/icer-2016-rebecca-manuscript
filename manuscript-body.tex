\section{Introduction}\label{introduction}

\subsection{We need more research on how software gets designed}\label{we-need-more-research-on-how-software-gets-designed}

In 2010, the journal \emph{Design Studies} devoted a special issue to studies of how professional software engineers design complex systems \cite{petre_editorial_2010}. The journal issue featured five different research perspectives on the same dataset: videos of three professional software engineering teams trying to design a traffic simulator. The existence of the journal issue, the monograph that followed it \cite{petre_software_2014} and the investigations contained therein, were motivated by what the issue's editors saw as a pressing need:

\begin{quote}
\emph{Not enough is known about the formative stages of software design.} The software engineering community has produced numerous design tools and languages, but, in practice, when software designers are first presented with a novel design problem, they more often than not will eschew these tools and languages. During formative design, software engineers spend a great deal of time engaging in creative, exploratory design thinking using pen and paper or a whiteboard---whether alone or in a small group. However, not enough is known about how software designers work in such settings. What do designers actually do during early software design? How do they communicate? What sorts of drawings do they create? \emph{What kinds of strategies do they software engineers apply in exploring the vast space of possible designs?} \cite{petre_editorial_2010} p.~533; emphasis mine
\end{quote}

Research in both the special issue \cite{petre_editorial_2010} and the monograph \cite{petre_software_2014} takes on a variety of challenges in the study of expert software design practice. One challenge, for example, is that of understanding how engineers process, prioritize, and cope with design requirements. Ball, Linden, Onarheim, and Christensen \cite{ball_design_2010} observed that engineers deploy mixed strategies, developing solutions breadth-first for easy problems and depth-first for more complex problems. They also found that the more complex a requirement became, the more likely engineers were to create speculative simulations (through talk and representations) about how a system might work to solve that problem \cite{ball_design_2010}.

Looking at the design sessions longitudinally, Baker and van der Hoek \cite{baker_ideas_2010} explored the shape and trajectory of how ideas generated in the design process develop and relate to one another. Those researchers found roughly a third of the ideas discussed in a typical design session ``were reiterations or rephrasings of previously stated ideas'' \cite{baker_ideas_2010}. While it is perhaps frustrating that ideas would be repeated so much, the authors argue such repetition can be viewed as a kind of continual revisitation to make sure a proposed design coheres:

\begin{quote}
Rather than representing a failure on the part of the designers, this repetition seems to be a necessary character of successful design sessions. Each time an idea is resurrected it is placed them sic in a new context, and compared to different aspects of the system. In this way, a concept of compatible, elegant design ideas is slowly converged upon. \cite{baker_ideas_2010}
\end{quote}
